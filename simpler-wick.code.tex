%%%%%%%%%%%%%%%%%%%%%%%%%%%%%%%%%%%%%%%%%%%%%%%%%%%%%%%%%%%%%%%%%%%%%%%%%%%%%%%%%
%
% Simpler-Wick
% Simpler Wick contractions in LaTeX
% Copyright (C) 2018  Joshua Ellis
%
%
% This work may be distributed and/or modified under the conditions of the LaTeX
% Project Public License, either version 1.3 of this license or (at your option)
% any later version.
%
% This work has the LPPL maintenance status `maintained'.
%
% The Current Maintainer of this work is Joshua Ellis.
%
% This program is distributed in the hope that it will be useful, but WITHOUT
% ANY WARRANTY; without even the implied warranty of MERCHANTABILITY or FITNESS
% FOR A PARTICULAR PURPOSE.  See the LaTeX Project Public License for more
% details.
%
%%%%%%%%%%%%%%%%%%%%%%%%%%%%%%%%%%%%%%%%%%%%%%%%%%%%%%%%%%%%%%%%%%%%%%%%%%%%%%%%

\usetikzlibrary{calc}

\pgfkeys{
  /simplerwick/.is family,
  /simplerwick,
  sep/.store in=\swick@sep,
  sep=3pt,
  offset/.store in=\swick@offset,
  offset=1em,
  line width/.store in=\swick@lwidth,
  line width=0.4pt}

% Works just like \@ifnextchar[, except for digits [1-9]
\def\swick@ifnextdigit#1#2{
    \@ifnextchar 1{#1}{
      \@ifnextchar 2{#1}{
        \@ifnextchar 3{#1}{
          \@ifnextchar 4{#1}{
            \@ifnextchar 5{#1}{
              \@ifnextchar 6{#1}{
                \@ifnextchar 7{#1}{
                  \@ifnextchar 8{#1}{
                    \@ifnextchar 9{#1}{#2}}}}}}}}}}

% Convert the digit to its representation as a word
\def\swick@translate#1{%
  \ifcase#1\or
  one\or
  two\or
  three\or
  four\or
  five\or
  six\or
  seven\or
  eight\or
  nine\fi
}

% Keeping track of which contraction is open or closed.
\newif\ifswick@if@one@
\newif\ifswick@if@two@
\newif\ifswick@if@three@
\newif\ifswick@if@four@
\newif\ifswick@if@five@
\newif\ifswick@if@six@
\newif\ifswick@if@seven@
\newif\ifswick@if@eight@
\newif\ifswick@if@nine@

% Shortcut to the ith 'if' variable
\def\swick@settrue@#1{
  \globaldefs=1
  \csname swick@if@\swick@translate{#1}@true\endcsname
  \globaldefs=0}
\def\swick@setfalse@#1{
  \globaldefs=1
  \csname swick@if@\swick@translate{#1}@false\endcsname
  \globaldefs=0}

% Returns true or false based on whether the ith.  Use as:
%
% \if\swick@cond@4
%   <if true>
% \else
%   <if false
% \fi
\def\swick@cond@#1{
  TT\fi
  \csname ifswick@if@\swick@translate{#1}@\endcsname}

% Check if *any* of the ifs is true.
\def\swick@cond@any#1#2{
  \if\swick@cond@1{#1}\else
    \if\swick@cond@2{#1}\else
      \if\swick@cond@3{#1}\else
        \if\swick@cond@4{#1}\else
          \if\swick@cond@5{#1}\else
            \if\swick@cond@6{#1}\else
              \if\swick@cond@7{#1}\else
                \if\swick@cond@8{#1}\else
                  \if\swick@cond@9{#1}\else
                    {#2}
  \fi\fi\fi\fi\fi\fi\fi\fi\fi}

% Set all ifs back to false.
\def\swick@cond@reset{
  \swick@setfalse@1
  \swick@setfalse@2
  \swick@setfalse@3
  \swick@setfalse@4
  \swick@setfalse@5
  \swick@setfalse@6
  \swick@setfalse@7
  \swick@setfalse@8
  \swick@setfalse@9
}

% Do nothing
\def\swick@ignore#1{}

% Delegate to the appropriate subfunction based on the next char:
% - 0: do nothing
% - [1-9]: Open the given contraction
% - anything else: Open contraction 1
\def\swick@smart{
  \@ifnextchar0{
    \swick@ignore
  }{
    \swick@ifnextdigit{
    \swick@smart@
  }{
    \swick@smart@1
  }}}

% Open or close the given contraction.
\def\swick@smart@#1#2{
  \if\swick@cond@#1
    \swick@end#1{#2}
  \else
    \swick@begin#1{#2}
  \fi}
  
% Open a contraction
\def\swick@begin#1#2{
  % swick@max keeps track of the tallest contraction yet.
  \ifnum\swick@max<#1
    \gdef\swick@max{#1}
  \fi
  \swick@settrue@#1
  \tikzexternaldisable
  \labelmathstyle{}
  \tikzpicture[remember picture, baseline=(swick-open#1.base)]
    \node[use as bounding box, inner sep=0pt, outer sep=0pt] (swick-open#1) {$\refmathstyle{} #2$};
  \endtikzpicture
  \tikzexternalenable
}

\def\swick@end#1#2{
  \swick@setfalse@#1
  \tikzexternaldisable
  \labelmathstyle{}
  \tikzpicture[remember picture, baseline=(swick-close#1.base)]
    \node[use as bounding box, inner sep=0pt, outer sep=0pt] (swick-close#1) {$\refmathstyle{} #2$};
  \endtikzpicture
  \tikz[remember picture, overlay]
    \draw[line width=\swick@lwidth] ($(swick-open#1.north) + (0, 3pt)$)
          -- ($(swick-open#1.base) + (0, \swick@offset) + #1*(0, \swick@sep)$) 
          -- ($(swick-close#1.base) + (0, \swick@offset) + #1*(0, \swick@sep)$) 
          -- ($(swick-close#1.north) + (0, 3pt)$);
  \tikzexternalenable}

% Start the \wick environment, checking optional arguments if needed.
\def\wick{
  \@ifnextchar[{
    \wick@
  }{
    \wick@[]
  }}

\def\wick@[#1]#2{
  \ifmmode
    \begingroup
    \pgfkeys{
        simplerwick,
        #1}
    % Define the variables and commands
    \swick@cond@reset
    \gdef\swick@max{0}
    \def\c{\swick@smart}
    % Here is the text
    #2
    % Add a vbox equal to max height
    \dimen0=\swick@sep
    \multiply\dimen0 by \swick@max
    \advance\dimen0 by \swick@offset
    \vbox to \dimen0{}
    % Check that every has been closed
    \swick@cond@any{
      \PackageWarning{simpler-wick}{%
        I have reached the end of \protect\wick\space with some unclosed
        contractions%
      }{}
    }{}
    \endgroup
  \else
    \PackageWarning{simpler-wick}{%
      \protect\wich\space has been called outside a math environment, this will
      be ignore%
    }
  \fi
}

\endinput

% Local Variables:
% TeX-master: "simpler-wick-doc"
% End:
