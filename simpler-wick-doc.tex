%%%%%%%%%%%%%%%%%%%%%%%%%%%%%%%%%%%%%%%%%%%%%%%%%%%%%%%%%%%%%%%%%%%%%%%%%%%%%%%%
%
% Simpler-Wick
% Simpler Wick contractions in LaTeX
% Copyright (C) 2018  Joshua Ellis
%
%
% This LaTeX file is free: you can redistribute it and/or modify it under the
% terms of the GNU General Public License as published by the Free Software
% Foundation, either version 3 of the License, or (at your option) any later
% version.
%
% This is distributed in the hope that it will be useful, but WITHOUT ANY
% WARRANTY; without even the implied warranty of MERCHANTABILITY or FITNESS FOR
% A PARTICULAR PURPOSE.  See the GNU General Public License for more details.
%
%%%%%%%%%%%%%%%%%%%%%%%%%%%%%%%%%%%%%%%%%%%%%%%%%%%%%%%%%%%%%%%%%%%%%%%%%%%%%%%%

%%%%%%%%%%%%%%%%%%%%%%%%%%%%%%%%%%%%%%%%%%%%%%%%%%%%%%%%%%%%%%%%%%%%%%%%%%%%%%%%
%% HEADER
%%%%%%%%%%%%%%%%%%%%%%%%%%%%%%%%%%%%%%%%%%%%%%%%%%%%%%%%%%%%%%%%%%%%%%%%%%%%%%%%

\def\pgfautoxrefs{1}
\documentclass[a4paper,final]{ltxdoc}

\usepackage{etex}

%% Formatting
%%%%%%%%%%%%%%%%%%%%%%%%%%%%%%%%%%%%%%%%%%%%%%%%%%%%%%%%%%%%%%%%%%%%%%%%%%%%%%%%

\usepackage[vmargin=4cm, hmargin=3cm]{geometry}  % Customize text width, page height, margins, etc.
% \usepackage{multicol}  % {multicols}{n} environment
% \usepackage{pdflscape} % {landscape} environment

%\usepackage{setspace} % Line spacing
%\singlespacing        % 1-spacing (default)
%\onehalfspacing       % 1,5-spacing
%\doublespacing        % 2-spacing

\usepackage{fontspec}  % Allows other fonts to be specified
\setmainfont{EB Garamond}[
  Contextuals={Alternate},
  Numbers=OldStyle,
]
\setmonofont{Inconsolata-g}[
  Scale=MatchLowercase,
]
\newfontface\swshape{EB Garamond 12 Italic}[Style=Swash]

\usepackage{microtype}   % Fine small typographical details
\usepackage{titlesec}    % Change section heading font
\titleformat*{\section}{\huge\scshape}
\titleformat*{\subsection}{\Large\scshape}
\titleformat*{\subsubsection}{\large\scshape}
\titleformat*{\paragraph}{\scshape}

\usepackage[symbol*,perpage]{footmisc}
\usepackage{enumitem}    % Easily customize lists
\setlist[description]{font=\normalfont\scshape}

%% Change the Table of Contents fonts
\usepackage{tocloft}
\renewcommand{\cftsecfont}{\normalfont\scshape}
\setlength\cftbeforesecskip{0pt}
\setcounter{tocdepth}{2}

%% Language
%%%%%%%%%%%%%%%%%%%%%%%%%%%%%%%%%%%%%%%%%%%%%%%%%%%%%%%%%%%%%%%%%%%%%%%%%%%%%%%%

\usepackage[UKenglish]{babel} % Set up the language
\usepackage{csquotes}
\usepackage{siunitx}          % Provides \SI{1}{\metre}, also \ang{90}.
\usepackage{textcomp}

%% Graphics & Figure
%%%%%%%%%%%%%%%%%%%%%%%%%%%%%%%%%%%%%%%%%%%%%%%%%%%%%%%%%%%%%%%%%%%%%%%%%%%%%%%%

\usepackage{graphicx}   % Allow graphics to be included
\usepackage{xcolor}     % Define and use colours
% \usepackage{subcaption} % Subfigures inside a figure

% Keep all pictures in the './img/' sub-directory.
\graphicspath{{./img/}}

%% Math Packages
%%%%%%%%%%%%%%%%%%%%%%%%%%%%%%%%%%%%%%%%%%%%%%%%%%%%%%%%%%%%%%%%%%%%%%%%%%%%%%%%

\usepackage{amsmath}   % The core math package
\usepackage{amssymb}   % Defines additional math fonts
\usepackage{mathtools} % Various extra maths functions
% \usepackage{cancel}    % Show cancellations with \cancel{}
% \usepackage{autonum}   % Only number referenced equations (must be loaded after cleverref)
\usepackage{dsfont}
\usepackage{simpler-wick}

\usepackage[cmintegrals,varg]{newtxmath} % Nice math with Garamond

%% Define \withnumber which forces the line to have number
\newcommand{\withnumber}{\refstepcounter{equation}\tag{\theequation}}

%% Allows page breaks in math (1 = avoid if possible, 4 = whenever)
%% Page breaks can be avoided at particular places by using \\*
\allowdisplaybreaks[2]

%% Tables
%%%%%%%%%%%%%%%%%%%%%%%%%%%%%%%%%%%%%%%%%%%%%%%%%%%%%%%%%%%%%%%%%%%%%%%%%%%%%%%%

\usepackage{array}     % New column types, including >{}x<{}
\usepackage{booktabs}  % Provides nicer horizontal lines
\usepackage{multirow}  % Allows cells to span multiple rows
%\usepackage{longtable} % Allows for tables to span multiple pages

%% Define the maths version of clr columns.
\newcolumntype{C}{>{\(}c<{\)}}
\newcolumntype{L}{>{\(}l<{\)}}
\newcolumntype{R}{>{\(}r<{\)}}

%% Other Packages
%%%%%%%%%%%%%%%%%%%%%%%%%%%%%%%%%%%%%%%%%%%%%%%%%%%%%%%%%%%%%%%%%%%%%%%%%%%%%%%%

\usepackage{listings}    % Code listings
\usepackage{minted}      % Use Pygments
\usepackage{fp}          % Floating point arithmetics
\usepackage{makeidx}     % Index creation
\usepackage{xr}          % Cross-referencing

\usepackage{hyperref}    % Automatically inserts hyperlinks.
\usepackage{cleveref}    % Use `\cref{}` to reference anything

\usepackage{autonum}

%% Other modifications
%%%%%%%%%%%%%%%%%%%%%%%%%%%%%%%%%%%%%%%%%%%%%%%%%%%%%%%%%%%%%%%%%%%%%%%%%%%%%%%%

% Modify the skip after each paragraph
\setlength{\parskip}{1ex plus 0.5ex minus 0.2ex}
\setlength{\parindent}{0pt}

\providecommand\href[2]{\texttt{#1}}
\providecommand\hypertarget[2]{\texttt{#1}}
\providecommand\hyperlink[2]{\texttt{#1}}

\newenvironment{example}
  {\VerbatimEnvironment
   \begin{VerbatimOut}{example.out}}
  {\end{VerbatimOut}%
   \vspace{1ex}%
   \setlength{\parindent}{0pt}%
   \fbox{\begin{minipage}{0.5\linewidth}%
     \inputminted[resetmargins]{latex}{example.out}%
   \end{minipage}%
   \hspace{0.05\linewidth}%
   \begin{minipage}{0.43\linewidth}%
     \input{example.out}%
   \end{minipage}%
   \vspace{1ex}}}

\let\origtexttt=\texttt
\def\texttt#1{{\def\textunderscore{\char`\_}\def\textbraceleft{\char`\{}\def\textbraceright{\char`\}}\origtexttt{#1}}}
\def\exclamationmarktext{!}
\def\atmarktext{@}

{
  \catcode`\|=12
  \gdef\pgfmanualnormalbar{|}
  \catcode`\|=13
  \AtBeginDocument{\gdef|{\ifmmode\pgfmanualnormalbar\else\expandafter\verb\expandafter|\fi}}
}

\makeatother

\makeindex

%% Define some slightly nicer colors
\definecolor{link-color}{RGB}{96 0 0}
\definecolor{cite-color}{RGB}{0 96 0}
\definecolor{file-color}{RGB}{0 0 96}
\definecolor{url-color}{RGB}{0 0 96}
\definecolor{link-border-color}{RGB}{255 159 159}
\definecolor{cite-border-color}{RGB}{159 255 159}
\definecolor{file-border-color}{RGB}{159 159 255}
\definecolor{url-border-color}{RGB}{159 159 255}

\hypersetup{
  pdfborder=0 0 0,
}

%% Document Information
%%%%%%%%%%%%%%%%%%%%%%%%%%%%%%%%%%%%%%%%%%%%%%%%%%%%%%%%%%%%%%%%%%%%%%%%%%%%%%%%

\makeatletter
\newcommand{\@keywords}{}
\newcommand{\keywords}[1]{\renewcommand{\@keywords}{#1}}

\newcommand{\@version}{}
\newcommand{\version}[1]{\renewcommand{\@version}{#1}}

\AtBeginDocument{
  \hypersetup{
    pdftitle={\@title},
    pdfauthor={\@author},
    pdfkeywords={\@keywords},
  }
}
\makeatother

\title{Simpler-Wick}
\version{1.0.0}
\author{Joshua \textsc{Ellis}}
\keywords{Wick contractions; TeX; LaTeX; TikZ; Simpler-Wick}



%%%%%%%%%%%%%%%%%%%%%%%%%%%%%%%%%%%%%%%%%%%%%%%%%%%%%%%%%%%%%%%%%%%%%%%%%%%%%%%%
%%     DOCUMENT
%%%%%%%%%%%%%%%%%%%%%%%%%%%%%%%%%%%%%%%%%%%%%%%%%%%%%%%%%%%%%%%%%%%%%%%%%%%%%%%%
\begin{document}

\begin{titlepage}
  \makeatletter
  \begin{center}
    \vspace*{1em}
    \tikz\node[scale=1.5]{%
      \color{gray}\Huge\ttfamily \char`\{\textcolor{red!75!black}{\@title}\char`\}};

    \vspace{0.5em}
    {\Large\bfseries Simpler Wick Contractions}

    \vspace{0.7em}
    {Version \@version \qquad \today}

    \vspace{1.3em}
    {by  \@author}
  \end{center}
  \makeatother

  \vfill

\begin{VerbatimOut}{example.out}
\(\wick{\c1Simp\c2le\c3r\ \c2Wick\ \c3Contractio\c1n}\)
\end{VerbatimOut}

  \begin{center}
    \tikzexternaldisable
    \tikz\node[scale=2]{\input{example.out}};
    \tikzexternalenable

    \begin{minipage}{0.65\linewidth}
      \inputminted{latex}{example.out}
    \end{minipage}
  \end{center}

  \vfill

  \tableofcontents
\end{titlepage}
%%%%%%%%%%%%%%%%%%%%%%%%%%%%%%%%%%%%%%%%%%%%%%%%%%%%%%%%%%%%%%%%%%%%%%%%%%%%%%%%
%% CONTENT
%%%%%%%%%%%%%%%%%%%%%%%%%%%%%%%%%%%%%%%%%%%%%%%%%%%%%%%%%%%%%%%%%%%%%%%%%%%%%%%%
\newpage
\section{Introduction}
\label{sec:introduction}

This package provides simple way of inserting Wick contractions.

If you have any suggestions or have found any bugs, please feel free to create a
new issue or pull request on the Github page:
\href{https://www.github.com/JP-Ellis/simpler-wick}{|https://www.github.com/JP-Ellis/simpler-wick|}.

\subsection{Installation}
\label{subsec:installation}

In order to use this as it is, simply download |simpler-wick.sty| and place it
in the same directory as your \TeX~file and include it using the usual
|\usepackage{simpler-wick}|.  Alternatively, it is also possible to install
|simpler-wick| system-wide by placing it inside \TeX's search path (which will
vary based on your operating system).  This package is also available through
\href{https:/ctan.org}{\textsc{ctan}}.


\section{Usage}
\label{sec:usage}

The package is imported by adding |\usepackage{simpler-wick}| to your preamble.
In your math environment, you can now use the |\wick| command in combination
with |\c|:

\begin{example}
  \begin{equation}
    \wick{\c\phi A \c\phi}
  \end{equation}
\end{example}

If you wish to have multiple contractions, then follow |\c| with a number
between 1 and 9;  the first occurrence of |\cN| will start the Wick contraction,
and the second occurrence of |\cN| will end it.  After you have ended a
contraction, |\cN| start another contraction.

\begin{example}
  \begin{equation}
    \wick{
      \c1 a \c2 b \c3 c \c1 a \c4 d \c1 e
      \c1 e \c1 a \c2 b \c3 c \c1 a
    }
  \end{equation}
\end{example}

The package has three options: |line width|, |sep| and |offset|.  |line width|
determines the thickness of the contraction lines; |sep| is the distance
separating each level; and |offset| is the base offset.  By default,
|linewidth=0.4pt|, |sep=3pt| and |offset=1em|, but they can be changed globally
by specifying them as package variables:

\begin{minted}{latex}
\usepackage[sep=5pt, offset=1.5em, line width=1pt]{simpler-wick}
\end{minted}

Or you can specify them as optional argument to |\wick|.  This is particularly
useful if you have some tall symbols within your Wick contraction:

\begin{example}
  \begin{equation}
    \wick[offset=2em]{\c\phi \int \frac{dx}{x} \c\phi}
  \end{equation}
\end{example}

\begin{example}
  \begin{equation}
    \wick[line width=1.5pt]{\c\phi A \c\phi}
  \end{equation}
\end{example}

\end{document}

%%% Local Variables:
%%% TeX-master: t
%%% End:
