%%%%%%%%%%%%%%%%%%%%%%%%%%%%%%%%%%%%%%%%%%%%%%%%%%%%%%%%%%%%%%%%%%%%%%%%%%%%%%%%
%
% Feynman Diagrams with TikZ
% Copyright (C) 2014  Joshua Ellis
%
% Allows Feynman diagrams to be used with TikZ.
%
%
%% This LaTeX file is free: you can redistribute it and/or modify it under the
% terms of the GNU General Public License as published by the Free Software
% Foundation, either version 3 of the License, or (at your option) any later
% version.
%
% This is distributed in the hope that it will be useful, but WITHOUT ANY
% WARRANTY; without even the implied warranty of MERCHANTABILITY or FITNESS FOR
% A PARTICULAR PURPOSE.  See the GNU General Public License for more details.
%
%%%%%%%%%%%%%%%%%%%%%%%%%%%%%%%%%%%%%%%%%%%%%%%%%%%%%%%%%%%%%%%%%%%%%%%%%%%%%%%%

%%%%%%%%%%%%%%%%%%%%%%%%%%%%%%%%%%%%%%%%%%%%%%%%%%%%%%%%%%%%%%%%%%%%%%%%%%%%%%%%
%% HEADER
%%%%%%%%%%%%%%%%%%%%%%%%%%%%%%%%%%%%%%%%%%%%%%%%%%%%%%%%%%%%%%%%%%%%%%%%%%%%%%%%

\def\pgfautoxrefs{1}
\documentclass[a4paper,final]{ltxdoc}

%% Formatting
%%%%%%%%%%%%%%%%%%%%%%%%%%%%%%%%%%%%%%%%%%%%%%%%%%%%%%%%%%%%%%%%%%%%%%%%%%%%%%%%

\usepackage{multicol}
\usepackage[
    a4paper,
    hmargin=2.25cm,
    vmargin=2.5cm,
    nohead
]{geometry} % Easily change margin sizes (and many more dimensions)

%\usepackage{setspace} % Line spacing
%\singlespacing        % 1-spacing (default)
%\onehalfspacing       % 1,5-spacing
%\doublespacing        % 2-spacing

%% Language
%%%%%%%%%%%%%%%%%%%%%%%%%%%%%%%%%%%%%%%%%%%%%%%%%%%%%%%%%%%%%%%%%%%%%%%%%%%%%%%%

\usepackage[UKenglish]{babel}
\usepackage[T1]{fontenc}
% \usepackage{microtype}
\usepackage{pifont}

%% Graphics & Figure
%%%%%%%%%%%%%%%%%%%%%%%%%%%%%%%%%%%%%%%%%%%%%%%%%%%%%%%%%%%%%%%%%%%%%%%%%%%%%%%%

\usepackage[pdftex]{graphicx}   % For loading graphic files

\usepackage{tikz}
% Set up PGF externalization
% Note that this requires the folder pgf-img to already exist
\usetikzlibrary{external}
\tikzexternalize[shell escape=-shell-escape, prefix=pgf-img/]
\tikzset{
    external/mode=list and make,
    external/system call={
        lualatex \tikzexternalcheckshellescape -halt-on-error -interaction=batchmode -jobname="\image" "\texsource" || rm "\image.pdf"},
}


%% Math Packages
%%%%%%%%%%%%%%%%%%%%%%%%%%%%%%%%%%%%%%%%%%%%%%%%%%%%%%%%%%%%%%%%%%%%%%%%%%%%%%%%

\usepackage{amsmath,amsfonts,amssymb} % The core packages for math
\usepackage{mathtools}                % Extra features
\usepackage{simpler-wick}

% Add tags to referenced lines only
\mathtoolsset{showonlyrefs,showmanualtags}
% Define \withnumber which forces a single line to have a number
\def\withnumber{\refstepcounter{equation}\tag{\theequation}}

% Allows page breaks in math (1 = avoid if possible, 4 = whenever)
% Page breaks can be avoided at particular places by using \\*
\allowdisplaybreaks[2]

%% Other Packages
%%%%%%%%%%%%%%%%%%%%%%%%%%%%%%%%%%%%%%%%%%%%%%%%%%%%%%%%%%%%%%%%%%%%%%%%%%%%%%%%

\usepackage{fancyvrb}
\usepackage{enumitem}    % Can customize {enumerate} and {itemize} lists
\usepackage{hyperref}    % Automatically inserts hyperlinks.
\usepackage{listings}    % Code listings
\usepackage{fp}          % Floating point arithmetics
\usepackage{minted}      % Use Pygments
\usepackage{makeidx}
\usepackage{xr}          % Cross-referencing


%% Other modifications
%%%%%%%%%%%%%%%%%%%%%%%%%%%%%%%%%%%%%%%%%%%%%%%%%%%%%%%%%%%%%%%%%%%%%%%%%%%%%%%%

\makeatletter
% Modify the skip after each paragraph
\setlength{\parskip}{1ex plus 0.5ex minus 0.2ex}
% \setlength{\headheight}{25.23pt}

\definecolor{link-color}{RGB}{0 0 75}
\definecolor{cite-color}{RGB}{75 0 25}
\definecolor{file-color}{RGB}{75 25 0}
\definecolor{url-color}{RGB}{75 25 0}
\definecolor{link-border-color}{RGB}{100 200 255}
\definecolor{cite-border-color}{RGB}{255 0 150}
\definecolor{url-border-color}{RGB}{255 150 0}

\hypersetup{
    pdftitle={simpler-wick: Simpler Wick contractions},
    pdfkeywords={wick contraction; TeX; LaTeX; ConTeXt; Tikz; pgf; simpler-wick},
    colorlinks=true,      % If colorlinks is false, a border is drawn instead
                          % which does not appear in print.  If it is true, the
                          % font is coloured and does appear in print.
    linkcolor=link-color,
    citecolor=cite-color,
    filecolor=file-color,
    urlcolor=url-color,
    linkbordercolor=link-border-color,
    citebordercolor=cite-border-color,
    urlbordercolor=url-border-color,
}

\providecommand\href[2]{\texttt{#1}}
\providecommand\hypertarget[2]{\texttt{#1}}
\providecommand\hyperlink[2]{\texttt{#1}}

\newenvironment{example}
  {\VerbatimEnvironment
   \begin{VerbatimOut}{example.out}}
  {\end{VerbatimOut}%
   \vspace{1ex}%
   \setlength{\parindent}{0pt}%
   \fbox{\begin{minipage}{0.5\linewidth}%
     \inputminted[resetmargins]{latex}{example.out}%
   \end{minipage}%
   \hspace{0.05\linewidth}%
   \begin{minipage}{0.4\linewidth}%
     \input{example.out}%
   \end{minipage}%
   \vspace{1ex}}}

\let\origtexttt=\texttt
\def\texttt#1{{\def\textunderscore{\char`\_}\def\textbraceleft{\char`\{}\def\textbraceright{\char`\}}\origtexttt{#1}}}
\def\exclamationmarktext{!}
\def\atmarktext{@}

{
  \catcode`\|=12
  \gdef\pgfmanualnormalbar{|}
  \catcode`\|=13
  \AtBeginDocument{\gdef|{\ifmmode\pgfmanualnormalbar\else\expandafter\verb\expandafter|\fi}}
}

\makeatother

\makeindex

%%%%%%%%%%%%%%%%%%%%%%%%%%%%%%%%%%%%%%%%%%%%%%%%%%%%%%%%%%%%%%%%%%%%%%%%%%%%%%%%
%%     DOCUMENT
%%%%%%%%%%%%%%%%%%%%%%%%%%%%%%%%%%%%%%%%%%%%%%%%%%%%%%%%%%%%%%%%%%%%%%%%%%%%%%%%
\begin{document}

\begin{center}
\vspace*{1em}
\tikz\node[scale=1.2]{%
  \color{gray}\Huge\ttfamily \char`\{\textcolor{green!50!black}{simpler-wick}\char`\}};

\vspace{0.5em}
{\Large\bfseries Simpler Wick Contractions}

\vspace{0.7em}
{Version 0.1.0 \qquad \today}

\vspace{1.3em}
{by  Joshua Ellis}
\end{center}

\vfill

\begin{VerbatimOut}{example.out}
\(\wick{\c1Simp\c2le\c3r\ \c2Wick\ \c3Contracti\c1{on}}\)
\end{VerbatimOut}

\begin{center}
  \tikz\node[scale=2]{\input{example.out}};
  
  \begin{minipage}{0.65\linewidth}
    \inputminted{latex}{example.out}
  \end{minipage}
\end{center}

\vfill

\tableofcontents
%%%%%%%%%%%%%%%%%%%%%%%%%%%%%%%%%%%%%%%%%%%%%%%%%%%%%%%%%%%%%%%%%%%%%%%%%%%%%%%%
%% CONTENT
%%%%%%%%%%%%%%%%%%%%%%%%%%%%%%%%%%%%%%%%%%%%%%%%%%%%%%%%%%%%%%%%%%%%%%%%%%%%%%%%
\newpage
\section{Introduction}
\label{sec:introduction}

This package provides simple way of inserting Wick contractions.

If you have any suggestions or have found any bugs, please feel free to create a
new issue or pull request on the Github page:
\href{https://www.github.com/JP-Ellis/simpler-wick}{|https://www.github.com/JP-Ellis/simpler-wick|}.

\subsection{Installation}
\label{subsec:installation}

This package is \emph{not} currently offered on
\href{https://www.ctan.org}{CTAN} as it is just a personal project of mine;
however, if enough people find it useful, I will look into making it available
through CTAN.

In order to use this as it is, simply download |simpler-wick.sty| and place it
in the same directory as your \TeX~file and include it using the usual
|\usepackage{simpler-wick}|.  Alternatively, it is also possible to install
|simpler-wick| system-wide by placing it inside \TeX's search path (which will
vary based on your operating system).


\section{Usage}
\label{sec:usage}

The package is imported by adding |\usepackage{simpler-wick}| to your preamble.
In your math environment, you now use the |\wick| command in combination with
|\c|:

\begin{example}
  \begin{equation}
    \wick{\c\phi A \c\phi}
  \end{equation}
\end{example}

If you wish to have multiple contractions, then follow |\c| with a number
between 1 and 9;  the first occurrence of |\cN| will start the Wick contraction,
and the second occurrence of |\cN| will end it.  After you have ended a
contraction, |\cN| start another contraction.

\begin{example}
  \begin{equation}
    \wick{
      \c1 a \c2 b \c3 c \c1 a \c4 d \c1 e
      \c1 e \c1 a \c2 b \c3 c \c1 a
    }
  \end{equation}
\end{example}

The package has two options: |sep| and |offset|.  |sep| is
the distance separating each level and |offset| is the base offset.  By default,
|\sep=3pt| and |\offset=1em|, but they can be changed globally by specifying
them as package variables:

\begin{minted}{latex}
\usepackage[sep=5pt, offset=1.5em]{simpler-wick}
\end{minted}

Or you can specify them as optional argument to |\wick|.  This is particularly
useful if you have some tall symbols within your Wick contraction:

\begin{example}
  \begin{equation}
    \wick[offset=2em]{\c\phi \int \frac{dx}{x} \c\phi}
  \end{equation}
\end{example}

\begin{example}
  \begin{equation}
    \wick[sep=10pt]{\c1\phi \c2\psi \c2\psi \c1\phi}
  \end{equation}
\end{example}

\end{document}

%%% Local Variables:
%%% TeX-master: t
%%% End:
